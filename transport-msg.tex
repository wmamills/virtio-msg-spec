\section{Virtio Over Messages}\label{sec:Virtio Transport Options / Virtio Over Messages}

The goal of the messages transport is to have commands communicated between the
frontend (driver) and backend (device) sides over a message based conduit
allowing to reach destination endpoints such as another VM, a TEE or a remote
compute engine using different communication channels.

The messages transport is designed to work efficiently between VMs on an hypervisor
based configuration or between heterogeneous systems by using messages to
reduce the number of context switches between the device and the driver sides
when each of them is in a different VM. In the same way, this is also reducing
cross system communications required in an heterogeneous system configuration.

\subsection{Messages Format}\label{sec:Virtio Transport Options / Virtio Over Messages / Messages Format}

The messages have the following properties:
\begin{itemize}
\item A message size is 40 bytes.
\item Most messages are driver/device initialization messages.
\item Messages are expected to be sent at low frequencies and quick round-trip is
  not necessary and should not impact global performances (as communication
  requiring performance should use virtqueues).
\end{itemize}

The virtio-MSG messages are encoded as following:

\begin{tabularx}{\textwidth}{|l|l|l|X|}
\hline
Name & Offset & Size (bytes) & Content \\
\hline \hline
VIRTIO_MSG_TYPE & 0 & 1 & Bit[0]: 0=Request, 1=Answer \newline Bit[1]: 0=virtio-msg, 1=bus-msg \newline Bit[2-7]: Reserved \\
\hline
VIRTIO_MSG_ID & 1 & 1 & Message ID \\
\hline
VIRTIO_MSG_DEV_ID & 2 & 2 & Device ID \\
\hline
VIRTIO_MSG_PAYLOAD & 4 & 36 & Payload \\
\hline
\end{tabularx}

\subsection{Errors Definition}\label{sec::Virtio Transport Options / Virtio Over Messages / Errors Definition}

The Virtio Message transport defines a list of error codes that are used in
the VIRTIO\_MSG\_ERROR message.

\begin{tabular}{|l|l|l|}
\hline
Error Code & Name & Description \\
\hline \hline
1  & EINVAL & Invalid parameter or feature \\
2  & ENOTSUPP & Unsupported function or feature \\
3  & EBUSY & Device or ressource unavailable \\
4  & EABORT & Request was aborted \\
5  & EIO & Input/Output error \\
6  & ENOMEM & No memory available \\
7  & EPERM & Permission denied \\
8  & EFAULT & Bad address \\
9  & ENODEV & No such device \\
\hline
\end{tabular}


\subsection{Messages Definition}\label{sec:Virtio Transport Options / Virtio Over Messages / Messages Definition}

The following table is listing the different Virtio-MSG messages and the sender
for each of them.

\begin{tabular}{|l|l|l|}
\hline
Name & ID & Sender \\
\hline
\hline
Reserved                      & 0x0 & \\
\hline
VIRTIO_MSG_ERROR              & 0x1  & Any    \\
\hline
VIRTIO_MSG_GET_DEVICE_INFO    & 0x2  & Driver \\
\hline
VIRTIO_MSG_GET_FEATURES       & 0x3  & Driver \\
\hline
VIRTIO_MSG_SET_FEATURES       & 0x4  & Driver \\
\hline
VIRTIO_MSG_GET_CONFIG         & 0x5  & Driver \\
\hline
VIRTIO_MSG_SET_CONFIG         & 0x6  & Driver \\
\hline
VIRTIO_MSG_GET_CONFIG_GEN     & 0x7  & Driver \\
\hline
VIRTIO_MSG_GET_DEVICE_STATUS  & 0x8  & Driver \\
\hline
VIRTIO_MSG_SET_DEVICE_STATUS  & 0x9  & Driver \\
\hline
VIRTIO_MSG_GET_VQUEUE         & 0xA  & Driver \\
\hline
VIRTIO_MSG_SET_VQUEUE         & 0xB  & Driver \\
\hline
VIRTIO_MSG_RESET_VQUEUE       & 0xC  & Driver \\
\hline
VIRTIO_MSG_EVENT_CONFIG       & 0x20 & Device \\
\hline
VIRTIO_MSG_EVENT_AVAIL        & 0x21 & Driver \\
\hline
VIRTIO_MSG_EVENT_USED         & 0x22 & Device \\
\hline
\end{tabular}

The following sections are giving more details of the usage of each message
and the encoding of the payload for the request and answer (when applicable).

\newcommand{\msgdef}[1]{\subsubsection{VIRTIO_MSG_#1}\label{sec:Virtio Transport Options / Virtio Over Messages / Messages Definition / VIRTIO_MSG_#1}}

\msgdef{ERROR}

The error message is used to signal to the driver or the device that a received
message is not valid or cannot be addressed. It can also be used to report an
internal error to the other entity.

The message contains a Virtio Message error code and the Message ID of the
request that triggerred the error (or 0 if the error was not in answer to a
specific request).

The message also contains a Data Type that can be set to STRING to use the
rest of the payload to give a NULL terminating string that can be used to carry
extra printable information.

This message can be received by the driver or the device and can be sent as an
answer to an other message by the driver or the device or any element between
them involved in the communication.

\begin{tabular}{|l|l|l|l|}
\hline
Type & Offset & Size (bytes) & Content \\
\hline \hline
Any     & 0 & 4 & Virtio Message Error Code \\
        & 4 & 1 & Message ID which generated an error or 0 if none \\
        & 5 & 1 & Data Type: 0 (Reserved), 1 (STRING), 2-255 (Implementation Defined) \\
        & 6 & 30 & Null Terminated String if Data Type is STRING else Implementation Defined \\
\hline
\end{tabular}

\msgdef{GET_DEVICE_INFO}

The Get device information message is used by the driver to retrieve the device
version, device ID and vendor ID.

This message is sent by the driver to the device and expects an answer from the
device.

\begin{tabular}{|l|l|l|l|}
\hline
Type & Offset & Size (bytes) & Content \\
\hline \hline
Request & 0 & 36 & Reserved (MBZ) \\
\hline
Answer & 0 & 4 & Device version \\
& 4 & 4 & Device ID \\
& 8 & 4 & Vendor ID \\
& 12 & 24 & Reserved (MBZ) \\
\hline
\end{tabular}

\msgdef{GET_FEATURES}

The Get features message is used to retrieve 256 bits of feature information
from the device. The driver request features at an index and the device answers
with the features bits at 256*index.

If there are no features available at the requested index but there are
features available after the index, the backend must return an all-zero
response.
If there are no features available at and after the requested index, the
backend must return an ERROR message to inform the frontend that it does not
need to scan features further.

This message is sent by the driver to the device and expects an answer from the
device.

\begin{tabular}{|l|l|l|l|}
\hline
Type & Offset & Size (bytes) & Content \\
\hline \hline
Request & 0 & 4 & Feature index \\
& 4 & 32 & Reserved (MBZ) \\
\hline
Answer & 0 & 4 & Feature index \\
& 4 & 32 & Feature data \\
\hline
\end{tabular}

\msgdef{SET_FEATURES}

The Set features message is used by the driver to configure the features of a
device. The driver configures 256 bits a time a given index. The device
provides in the answer the actual features value at the index after having
configured the bit requested giving an opportunity for the driver to detect if
some of the bits it tried to set were not accepted.

This message is sent by the driver to the device and expects an answer from the
device.

\begin{tabular}{|l|l|l|l|}
\hline
Type & Offset & Size (bytes) & Content \\
\hline \hline
Request & 0 & 4 & Feature index \\
& 4 & 32 & Feature data \\
\hline
Answer & 0 & 4 & Feature index \\
& 4 & 32 & Feature data \\
\hline
\end{tabular}

\msgdef{GET_CONFIG}

The Get configuration message is used by the driver to retrieve a part of the
configuration values of the device. The driver can request up to 256 bits at
a given offset in the device configuration space.

This message is sent by the driver to the device and expects an answer from the
device.

\begin{tabular}{|l|l|l|l|}
\hline
Type & Offset & Size (bytes) & Content \\
\hline \hline
Request & 0 & 3 & Configuration offset \\
& 3 & 1 & Number of bytes (1-32) \\
& 4 & 32 & Reserved (MBZ) \\
\hline
Answer & 0 & 3 & Configuration offset \\
& 3 & 1 & Number of bytes (1-32) \\
& 4 & 32 & Configuration data \\
\hline
\end{tabular}

\msgdef{SET_CONFIG}

The Set configuration message is used by the driver to modify a part of the
configuration values of the device. The driver can modify up to 256 bits at a
given offset in the device configuration. The device answers with the actual
configuration value at the offset after the modification to give a chance to the
driver to detect changes that have not been accepted by the device.

This message is sent by the driver to the device and expects an answer from the
device.

\begin{tabular}{|l|l|l|l|}
\hline
Type & Offset & Size (bytes) & Content \\
\hline \hline
Request & 0 & 3 & Configuration offset \\
& 3 & 1 & Number of bytes (1-32) \\
& 4 & 32 & Configuration data \\
\hline
Answer & 0 & 3 & Configuration offset \\
& 3 & 1 & Number of bytes (1-32) \\
& 4 & 32 & Configuration data \\
\hline
\end{tabular}

\msgdef{GET_CONFIG_GEN}

The Get configuration generation message is used by the driver to retrieve the
device configuration atomicity value.

TODO: need help to have a complete description here.

This message is sent by the driver to the device and expects an answer from the
device.

\begin{tabular}{|l|l|l|l|}
\hline
Type & Offset & Size (bytes) & Content \\
\hline \hline
Request & 0 & 36 & Reserved (MBZ) \\
\hline
Answer & 0 & 4 & Atomicity value \\
& 4 & 32 & Reserved (MBZ) \\
\hline
\end{tabular}

\msgdef{GET_DEVICE_STATUS}

The Get device status message is used by the driver to retrieve the device
status fields.

This message is sent by the driver to the device and expects an answer from the
device.

\begin{tabular}{|l|l|l|l|}
\hline
Type & Offset & Size (bytes) & Content \\
\hline \hline
Request & 0 & 36 & Reserved (MBZ) \\
\hline
Answer & 0 & 4 & Device status \\
& 4 & 32 & Reserved (MBZ) \\
\hline
\end{tabular}

\msgdef{SET_DEVICE_STATUS}

The Set device status message is used by the driver to modify the device status
fields. Writing a status of 0 triggers a device reset.

This message is sent by the driver to the device and expects an answer from the
device.

\begin{tabular}{|l|l|l|l|}
\hline
Type & Offset & Size (bytes) & Content \\
\hline \hline
Request & 0 & 4 & Device status \\
& 4 & 32 & Reserved (MBZ) \\
\hline
Answer & 0 & 36 & Reserved (MBZ) \\
\hline
\end{tabular}

\msgdef{GET_VQUEUE}

The Get vqueue message is used to retrieve the maximum virtqueue size and
information about the vqueue if it was already configured (all information are
0 if the virtqueue was not configured).

This message is sent by the driver to the device and expects an answer from the
device.

\begin{tabular}{|l|l|l|l|}
\hline
Type & Offset & Size (bytes) & Content \\
\hline \hline
Request & 0 & 4 & Virtqueue index \\
& 4 & 32 & Reserved (MBZ) \\
\hline
Answer & 0 & 4 & Virtqueue index \\
& 4 & 4 & Max virtqueue size \\
& 8 & 4 & Virtqueue size \\
& 12 & 8 & Descriptor address \\
& 20 & 8 & Driver address \\
& 28 & 8 & Device address \\
\hline
\end{tabular}

\msgdef{SET_VQUEUE}

The Set vqueue message is used to configure a virtqueue.
Setting the virtqueue size to 0 is disabling the virtqueue without modifying
the rest of the parameters (those should be ignored by the device).
If a driver needs to complete reset a virtqueue, the RESET\_VQUEUE message
should be used instead.

This message is sent by the driver to the device and expects an answer from the
device.

\begin{tabular}{|l|l|l|l|}
\hline
Type & Offset & Size (bytes) & Content \\
\hline \hline
Request & 0 & 4 & Virtqueue index \\
& 4 & 4 & Reserved (MBZ) \\
& 8 & 4 & Virtqueue size \\
& 12 & 8 & Descriptor address \\
& 20 & 8 & Driver address \\
& 28 & 8 & Device address \\
\hline
Answer & 0 & 4 & Virtqueue index \\
& 4 & 4 & Reserved (MBZ) \\
& 8 & 4 & Virtqueue size \\
& 12 & 8 & Descriptor address \\
& 20 & 8 & Driver address \\
& 28 & 8 & Device address \\
\hline
\end{tabular}

\msgdef{RESET_VQUEUE}

The Reset vqueue message is used to disable and reset a virtqueue.

This message is sent by the driver to the device and expects an answer from the
device.

\begin{tabular}{|l|l|l|l|}
\hline
Type & Offset & Size (bytes) & Content \\
\hline \hline
Request & 0 & 4 & Virtqueue index \\
& 4 & 32 & Reserved (MBZ) \\
\hline
Answer & 0 & 36 & Reserved (MBZ) \\
\hline
\end{tabular}

\msgdef{EVENT_CONFIG}

The Event config message is sent by the device to signal to the driver that one
or several values in the device configuration have changed. The message
contains the current device status (as encoded in the GET\_DEVICE\_STATUS
answer) and optionally the part of the configuration that has been modified. If
this is not provided, the driver will have to use the GET\_CONFIG to retrieve
the configuration and discover what has changed.

This message is sent by the device to the driver and does not expect any
answer.

\begin{tabular}{|l|l|l|l|}
\hline
Type & Offset & Size (bytes) & Content \\
\hline \hline
Request & 0 & 4 & Device status \\
& 4 & 3 & Configuration offset \\
& 7 & 1 & Number of bytes (1-16) \\
& 8 & 16 & Configuration data \\
& 24 & 12 & Reserved (MBZ) \\
\hline
\end{tabular}

\msgdef{EVENT_AVAIL}

The Event available message is sent by the driver to the device to signal that
some data are available to process in a virtqueue. If the NOTIFICATION\_DATA
was negotiated, the message can also include more details on what is actually
available.

This message is sent by the driver to the device and does not expect any
answer.

\begin{tabular}{|l|l|l|l|}
\hline
Type & Offset & Size (bytes) & Content \\
\hline \hline
Request & 0 & 4 & Virtqueue index \\
& 4 & 4 & Next offset \\
& 8 & 4 & Next wrap \\
& 12 & 24 & Reserved (MBZ) \\
\hline
\end{tabular}

\msgdef{EVENT_USED}

The event used message is sent by the device to the driver to signal that some
data is available or has been processed in the a virtqueue.

TODO: is there a case for offset/wrap in this one ?

This message is sent by the device to the driver and does not expect any
answer.

\begin{tabular}{|l|l|l|l|}
\hline
Type & Offset & Size (bytes) & Content \\
\hline \hline
Request & 0 & 4 & Virtqueue index \\
& 4 & 32 & Reserved (MBZ) \\
\hline
\end{tabular}

\subsection{Message Bus}\label{sec:Virtio Transport Options / Virtio Over Messages / Message Bus}

The Message Bus acts as a communication channel between two endpoints,
providing a standardized abstraction to exchange messages.

The Message Bus transmits the messages between the frontend and the backend in
both directions.

A system can have several Message Buses and it shall respect the following
principles:

\begin{itemize}
\item One Message Bus has a backend and a frontend side located in 2 different
      endpoints (an endpoint can be a VM, a secure Partition or a remote OS).
\item One Message Bus has one or more devices on the backend side handled by
      drivers on the frontend side.
\item An endpoint can be connected to several Message Buses.
\end{itemize}

A Message Bus implementation must provide interfaces to transmit and receive
messages with the other side.

While the Bus implementation and how communication of the messages is done
between the devices and the drivers is completely out of the scope of the
specification, this chapter provides a set of standard Bus messages to help
standardizing as much as possible a part of the bus implementation.

\subsubsection{Discovery}\label{sec:Virtio Transport Options / Virtio Over Messages / Message Bus / Discovery}

The Message Bus is responsible of listing the available buses and to enumerate
all devices available on each bus.

The frontend may use a predefined virtio devices ID list (static way), or may
use the BUS_MSG_GET_DEVICES message to discover devices IDs available on the
backend (dynamic way), or a combination of the two methods.

The backend may use a predefined virtio devices ID list that can be provided to
the driver on BUS_MSG_GET_DEVICES request, or may announce device IDs using
the BUS_MSG_DEVICE_ADDED and BUS_MSG_DEVICE_REMOVED messages, or a combination
of the two methods.

\subsubsection{Message Format}\label{sec:Virtio Transport Options / Virtio Over Messages / Message Bus / Message Format}

The Bus messages are transferred using the \emph{bus-msg} type in
VIRTIO_MSG_TYPE and have the following format:

\begin{tabularx}{\textwidth}{|l|l|l|X|}
\hline
Name & Offset & Size (bytes) & Content \\
\hline \hline
VIRTIO_MSG_TYPE & 0 & 1 & Bit[0]: 0=Request, 1=Answer \newline Bit[1]: 1=bus-msg \newline Bit[2-7]: Reserved \\
\hline
VIRTIO_MSG_ID & 1 & 1 & Message ID \\
\hline
VIRTIO_MSG_PAYLOAD & 2 & 38 & Payload \\
\hline
\end{tabularx}

\subsubsection{Message Definition}\label{sec:Virtio Transport Options / Virtio Over Messages / Message Bus / Message Definition}

The following table is listing the different Virtio-MSG Bus messages and the
sender for each of them.

\begin{tabular}{|l|l|l|}
\hline
Name & ID & Sender \\
\hline
\hline
Reserved                    & 0x0  &        \\
\hline
BUS_MSG_ERROR               & 0x1  & Any    \\
\hline
BUS_MSG_GET_DEVICES         & 0x2  & Driver \\
\hline
BUS_MSG_DEVICE_ADDED        & 0x3  & Device \\
\hline
BUS_MSG_DEVICE_REMOVED      & 0x4  & Device \\
\hline
BUS_MSG_PING                & 0x5  & Any    \\
\hline
BUS_MSG_RESET               & 0x6  & Driver  \\
\hline
\end{tabular}

Bus message ID 6-127 are reserved for future use in the specification and
message ID 128-255 can be used for implementation specific needs.

The following sections are giving more details of the usage of each message
and the encoding of the payload for the request and answer (when applicable).

\newcommand{\busdef}[1]{\paragraph{BUS_MSG_#1}\label{sec:Virtio Transport Options / Virtio Over Messages / Message Bus / Message Definition / BUS_MSG_#1}}

\busdef{ERROR}

The Error message is used by one side of the bus to inform the other side that
one message could not be processed correctly. The message contains a Virtio
Messsage error code.

This message shall be used for Bus errors not specific to a particular device.
In case of a device specific error not impacting the complete bus, the
VIRTIO_MSG_ERROR shall be used.

As for the VIRTIO_MSG_ERROR, this message supports to have a NULL terminating
String that can be used to give more information printable to the user.

This message can be received by the frontend or the backend side and can be
send as answer to an other message or directly to signal an error.

\begin{tabular}{|l|l|l|l|}
\hline
Type & Offset & Size (bytes) & Content \\
\hline \hline
Any     & 0 & 4 & Virtio Message Error Code \\
        & 4 & 1 & Message ID which generated an error or 0 if none \\
        & 5 & 1 & Data Type: 0 (Reserved), 1 (STRING), 2-255 (Implementation Defined) \\
        & 6 & 32 & Null Terminated String if Data Type is STRING else Implementation Defined \\
\hline
\end{tabular}

\busdef{GET_DEVICES}

The Get Devices message is used by the frontend to retrieve the IDs of the
devices available on the backend of the bus.

The frontend requests the availability of a group of 256 device IDs and the
backend returns a bitmap indicating which device IDs are available at the
requested device ID offset.

A request at offset X gives the availability of device IDs X*32 to (X+1)*32-1.

The backend also returns a next offset indicating to the frontend what device
ID offset should be requested next. This is to reduce the number of messages
required to scan all IDs when the backend has device IDs far from each other.

This message shall be used in the following way:
\begin{itemize}
\item frontend request device IDs at offset X.
\item backend answers with the availability of device IDs X*32 to (X+1)*32-1
  and indicates to the frontend which offset to request after using the next
  offset.
\end{itemize}
The frontend starts by doing a request with offset 0 and then iterates using
the next offset returned by the backend until the backend gives a next offset
of 0.

This message is sent by the frontend side of the bus to the backend side and
expects an answer from the backend side.

\begin{tabular}{|l|l|l|l|}
\hline
Type & Offset & Size (bytes) & Content \\
\hline \hline
Request & 0 & 2 & Offset \\
        & 2 & 34 & Reserved (MBZ) \\
\hline
Answer & 0 & 2 & Offset \\
       & 2 & 2 & Next offset or 0 if no devices after Offset \\
       & 4 & 2 & Reserved (MBZ) \\
       & 6 & 32 & Bit[n]: Device[offset*32 + n] not available(0)/available(1) \\
\hline
\end{tabular}

\busdef{DEVICE_ADDED}

This message is sent by the backend of the bus to signal to the frontend that
a new device appeared on the bus.

This message is sent by the backend of the bus and does not expect any answer.

\begin{tabular}{|l|l|l|l|}
\hline
Type & Offset & Size (bytes) & Content \\
\hline \hline
Request & 0 & 2 & Device ID \\
        & 2 & 34 & Reserved (MBZ) \\
\hline
\end{tabular}

\busdef{DEVICE_REMOVED}

This message is sent by the backend of the bus to signal to the frontend that
a device was removed from the bus.

This message is sent by the backend of the bus and does not expect any answer.

\begin{tabular}{|l|l|l|l|}
\hline
Type & Offset & Size (bytes) & Content \\
\hline \hline
Request & 0 & 2 & Device ID \\
        & 2 & 34 & Reserved (MBZ) \\
\hline
\end{tabular}

\busdef{PING}

The ping message can be sent by the backend or the frontend side of the bus
to check that the other side is responding properly.

This message can be used by either side to monitor that the other side of the
bus is available and functional.

The message contains a 32bit data field with a custom value that must be copied
back by the receiver in its answer.

\begin{tabular}{|l|l|l|l|}
\hline
Type & Offset & Size (bytes) & Content \\
\hline \hline
Request & 0 & 4 & Data \\
        & 4 & 32 & Reserved (MBZ) \\
\hline
Answer & 0 & 4 & Request Data \\
       & 4 & 32 & Reserved (MBZ) \\
\hline
\end{tabular}

\busdef{STATUS}

This message can be sent by the backend or the frontend to signal to the other
side that its own status has changed which required the other side to take
global actions.

For example this message can be send by the frontend when it starts to
request the backend to reset the bus and all devices on it or when the backend
stops to signal to the backend to shutdown everything on the bus.

The backend can use it to signal to the frontend that for some reason it will
shutdown which shall trigger the frontend to disable all drivers.

The message contains a New state field which gives to the receiving the new
state the sender is going into for which the following values are defined:
\begin{itemize}
\item 0: Reset.
\item 1: Shutdown.
\item 2: Suspend.
\item 4: Resume.
\item 5-127: Reserved.
\item 128-255: Implementation Defined.
\end{itemize}

This message can be sent by the frontend or the backend side of the bus and
does not expect any answer.

\begin{tabular}{|l|l|l|l|}
\hline
Type & Offset & Size (bytes) & Content \\
\hline \hline
Request & 0 & 1 & New state \\
        & 1 & 35 & Reserved (MBZ) \\
\hline
\end{tabular}

